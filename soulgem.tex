\documentclass[10pt, conference, compsocconf]{IEEEtran}

\usepackage{comment}
\usepackage{amsmath}
\usepackage{amsfonts}
\usepackage{amssymb}
\usepackage{array}
\usepackage{booktabs}
\usepackage{capt-of}
\usepackage{colortbl}
\usepackage{graphicx}
\usepackage{multirow}
\usepackage{subfig}
%\DeclareCaptionType{copyrightbox}
\usepackage{pifont}
\usepackage[latin1]{inputenc}
\usepackage{times}
\usepackage{url}
\usepackage{boxedminipage}
\usepackage{xspace}
\usepackage{sepnum}
\usepackage{cite}
\usepackage{fancyhdr}
\usepackage{rotating}

\begin{document}
\title{Setting the Stage for Understanding the Main Roles in Software Organizations}

\author{\IEEEauthorblockN{Jorge Aranda, Adrian Schr\"{o}ter, Daniela Damian, and Margaret-Anne Storey}
\IEEEauthorblockA{Department of Computer Science\\
University of Victoria\\
Victoria, Canada\\
Email: \{jaranda, schadr, danielad, mstorey\}@uvic.ca}
}

\maketitle


\begin{abstract}

Software professionals divide their work by roles, but to date, researchers have not examined nor questioned the nature of these roles. To address this, we formulate a theoretical framework based on sociological and organizational research, in which roles are defined as sets of expectations placed on the members of the software organization.

The paper also reports on an empirical case study of two software organizations, which was informed and inspired by this theoretical framework. In our study, we found that roles were inconsistent within and between the organizations we studied, and that role definitions are shaped by institutional, historical, structural, and personal factors. Furthermore, role definitions evolve through organizational growth and through mismatches between expectations and performance, and people enacting different roles vary in their selection of communication media.

Our contributions should help researchers to frame better studies of division of labour in software groups and to avoid unwarranted abstractions of professional software roles. They also provide practitioners with a conceptual tool to reason about social structure and organizational design.

\end{abstract}

\begin{IEEEkeywords}
software organizations; social structure; roles
\end{IEEEkeywords}

%%%%%%%%%%%%%%%%%%%%%%%%%%%%%%%%%%%%%%%%%%%%%%%%%%%%%%%%%%%%%%%%%%%%

\section{Introduction}

People divide their labour when producing software systems: certain people write code, others test it, and some ensure that the product satisfies a real need. Researchers and practitioners abstract these responsibilities into a number of role labels, such as ``developer'', ``tester'', and ``product manager'', and we use these abstractions to refer to the set of activities that we expect them to carry out. These labels are common in academic literature, as well as in textbooks, books for practitioners, human resources postings, blog posts, and everyday conversation.

Despite their widespread use, we have no clear indication that the people using these labels refer to the same definitions. For instance, what does a ``developer'' actually do in order to fulfill her responsibilities? What bounds her activity? And would the answer be generalizable throughout the software industry?

In some domains, especially ones that are more mature or critical, the division of labour is fairly formal and clear. In the medical field, surgeons, anesthesiologists, and nurses generally know what to expect from each other, in and out of the operating room. Knowing that someone is a nurse will bring out a set of behavioural expectations that have a good probability of being shared across the board. But although we assign labels just as effortlessly as in the medical field, we do not realize that we may be talking about different things, based on our varying assumptions, experiences, and cultures.

This ontological drift \cite{Robinson1991} in our conceptualization of software development roles has yet to be studied in depth, and it has unexplored repercussions for researchers and practitioners alike. Poorly defined roles may lead to dysfunction in a project team, and thus negatively impact the quality of the software. Researchers may prescribe that practices be carried out by one role without realizing that the role is usually shared and spread out among a group of people. Practitioners may not know how a newcomer's experience as a developer in a different organization will map into their own, or how to train him \cite{Dagenais2010}. Their communication tools may fail to be adopted if they satisfy the needs of some roles but not of others. For an abstraction as heavily used as roles in software organizations, a closer look should prove beneficial.

In this paper, we do not provide a rigid definition of roles in software organizations; in fact, we argue that such a task has a poor prospect of success. Instead, (a) we examine the concepts of roles, positions, and expectations in software groups, providing a theoretical framework based on interdisciplinary research, (b) we report the findings from a case study at two software organizations in which we used this framework to ground our observations and interviews, and (c) we discuss the implications for research and practice from the framework and from our findings.



%%%%%%%%%%%%%%%%%%%%%%%%%%%%%%%%%%%%%%%%%%%%%%%%%%%%%%%%%%%%%%%%%%%%%%%%%%%

\section{Related Work}

Sociologists and organizational scientists have explored the topic of professional and social roles for a long time. Biddle \cite{Biddle1986} provides a good point of entry into the theoretical work in this area. His overview covers five main variations of theoretical thinking: \emph{functional} role theory (roles are functions in a stable social system), \emph{symbolic interactionist} role theory (roles evolve through social interaction, norms only provide broad imperatives), \emph{structural} role theory (roles are patterns of behaviour shared among people in similar positions in the social structure), \emph{organizational} role theory (roles are mainly set by social positions in preplanned, rational settings), and \emph{cognitive} role theory (which focuses on determining which conditions give rise to expectations, how to measure expectations, and what impact do they have on social conduct). Although there are significant philosophical and methodological differences between role theories, they are compatible on many points. As Biddle points out, ``most versions of role theory presume that expectations are the major generators of roles, that expectations are learned through experience, and that persons are aware of the expectations they hold.''

A similar conceptualization comes from Turner \cite{Turner1956}, who states that roles are sets of norms, where a norm is an expected or appropriate behaviour that should be consistent. A role ``is made up of all those norms which are thought to apply to a person occupying a given position.'' Scott \cite{Scott2008} concurs, pointing out that roles are normative expectations of the appropriate goals and activities for particular individuals, and that roles can be formally constructed or emerge informally through interaction.

A good analysis of role evolution comes from Nicholson \cite{Nicholson1984}. He studied how people adapt to new work roles, and determined there are four main ways in which they do so, based on how much they change themselves and the environment: replication (neither the person nor the environment change much), absorption (personal development), determination (role development), and exploration (the person develops both herself and the meaning of her role).

In the software engineering literature, researchers discuss roles frequently, in both prescriptive and descriptive efforts. Chief Programmer Teams \cite{Baker1972} are an early example of a prescriptive approach. Process definitions tend to include role prescriptions: Scrum \cite{Schwaber2001}, for instance, calls for a Scrum Master, a Product Owner, and a number of developers. Practices also prescribe roles, but in a more minute detail and, usually, for a shorter term. Thus in a Fagan inspection \cite{Fagan1976}, we have a Moderator, a Reader, a Coder, and several Reviewers.

Recent work discusses roles in interesting ways, but without a solid or consistent theoretical framework. Hoda \emph{et al.}\ \cite{Hoda2010} identified six ``informal roles'' that team members adopt in self-organizing teams. The authors distinguish between roles and the people that enact them, which is a distinction we advocate for in this paper. But several of the roles they identify (such as Champion and Promoter) are too fine-grained, and they fall under the category of single expectations in our framework below---expectations that are bundled with others to produce a richer, multi-faceted role.

Moe \emph{et al.} \cite{Moe2010} use Dickinson and McIntyre's \cite{Dickinson1997} teamwork model to provide such a set of expectations for members of self-managing teams. This is certainly useful, although it cannot be used as a basis for our work, as it does not allow us to study how or why differences in expectations and specializations within teams arise.

Other researchers have attempted to study roles structurally, through social network analysis. For instance, Toral \emph{et al.}\ \cite{Toral2010} studied ``middle-men'' or ``brokers'', and Ricca and Marchetto \cite{Ricca2010} examine the role of ``heroes'', in open source projects. Their structural analysis allows them to scale to large projects and electronic repositories. To our knowledge, more sophisticated analyses of roles from a structural role theory perspective, such as the sociological blockmodelling work of Breiger and Pattison \cite{Breiger1986}, has not been carried out in our domain.

We can also find rich descriptions of some roles and role interactions in software development. Martin \emph{et al.}\ \cite{Martin2007} report a detailed picture of the testing role carried out by the developers in a software organization. F{\ae}gri \emph{et al.} \cite{Faegri2010} provide a comprehensive report of the consequences of role evolution in a case in which developers attempted to rotate the responsibilities of customer support, and Jensen and Scacchi \cite{Jensen2007} describe the migrations in role responsibilities in several open source projects. De Souza \emph{et al.}\ \cite{deSouza2003} look at inter-role communication, and describe that problem reports are used as boundary objects \cite{Bowker1999} between people with different roles, while Marczak and Damian \cite{Marczak2011} explore the backchannel communication between people with different roles.

Finally, Litecky \emph{et al.}\ \cite{Litecky2010} report on the positions offered and the skills required in the computing job market. Their analysis is useful for a study of roles in software organizations as it provides us with an indication of the roles that human resources professionals consider important in their groups.



%%%%%%%%%%%%%%%%%%%%%%%%%%%%%%%%%%%%%%%%%%%%%%%%%%%%%%%%%%%%%%%%%%%%%%

\section{Theoretical Framework}

The following framework is distilled from the work of sociologists and organizational scientists. To our knowledge this is the first attempt at analyzing these concepts in the context of software development work.


\subsection{Roles, Positions, and Labels}

A \textbf{\emph{role}} is a set of behavioural expectations placed on a person given their position, status, history, or preference \cite{Turner1956,Gordon1976,Rizzo1970}. This entails that roles are \emph{social constructions} \cite{Berger1967}. For example, a ``developer'' is whatever the relevant social group determines that a developer should be. There is no prototypical developer. Upon learning that a colleague is a developer, her peers place a number of expectations on her. Perhaps they expect her to spend as much time as possible writing production code, to be responsive to user requests for assistance, to validate the code produced by other developers, or to self-manage. Some colleagues may have no expectations other than that the developer should strive for the well-being of the organization. Other colleagues, especially if they interact with the developer on a regular basis, may have more expectations that may be far more concrete, such as to adhere to the specified API or to put certain test hooks in place. People outside the formal organization, but connected to it, may have further expectations. And the developer herself, the target of all these expectations, likely has a number of additional expectations of her own conduct: the activities she feels she should be doing to play her part in the organization.

This collection of expectations define the role. The resulting set may be straightforward or conflictive (a phenomenon called \emph{role strain} \cite{Goode1960}), and it may be impossible to satisfy it in full. Since the expectations arise from the group, there is no guarantee that they will match those of other groups; the meaning of a ``developer'' role in one organization may differ from that in another organization, and even within teams in an organization the semantic differences may be profound.

A \textbf{\emph{position}} or a \textbf{\emph{title}}, in turn, is an official statement of a person's goals and responsibilities. Positions and their responsibilities are often the explicit articulation of what the managers expect from their employees. A role is not a position \cite{Turner1956}, though they often carry the same labels. One can play an architectural role and hold the position of Software Architect. A person's title plays a part in determining what she will feel compelled to do, and what she believes she is expected to do, but it is only one element of her role. For instance, one could enact an architectural role without holding an Software Architect position, and one could hold a Software Architect position without performing any of the activities that others would expect from an architect.

As indicated above, we often use \textbf{\emph{labels}} to refer to roles. A label helps us convey the essence of the role, but some roles have ill-fitting labels, and others have no labels or easy ways to refer to them in concrete terms. We must remember that neither the label nor the formal position completely define the role.


\subsection{Institutionalized Roles}

Culture plays an important part in setting and steering our expectations. In some domains, cultural norms are strong enough to establish an interpretation of a role throughout the population (the role has been institutionalized \cite{DiMaggio1991}). In these cases, a combination of intensive training, ethical codes, professional societies, and familiarization lead us to expect certain behaviours and activities from a role. In our domain, these institutional forces allow us to make some general assumptions that we can reasonably expect will hold in most cases. For example, we might be surprised if a tester did not spend at least part of his time preparing or running tests. And so, thanks to this process of institutionalization, a newcomer to the test department will likely assume that he is expected to prepare and run tests, his peers and supervisors will indeed place these expectations upon him, and this is what he will spend his time doing, thereby reinforcing the general idea of what testers do.

However, the forces of the larger culture have not yet enforced a singular view of what the different roles in a software organization consist of in any useful detail, or even which roles exist. As the evidence of our case study shows, significant disagreements exist even within the same software organizations. Determining the reasons for this disagreement is beyond the goals of this paper; we speculate that the relative immaturity, the wide scope, and the low barriers of entry to the software industry contribute to the phenomenon. We make no judgment regarding the convenience of this lack of uniformity in role definitions, but we believe it should be explored in greater detail.


\subsection{Role Evolution}

Group members do not always act according to the expectations placed on them, for several reasons. Their performance could fail to meet expectations because of unrealistic expectations, unskilled, careless, or unaware performers, or conflicting expectations. Their performance could also exceed expectations due to the performer's initiative, talent, or background. In any case, if disagreements recur over time, organization members undergo a shift in their expectations: they learn to expect the ``new normal''. They adapt their expectations, and since roles are conformed by these expectations, role definitions evolve as a result of past performance, despite their institutional status.

Another reason for role evolution arises from changes in the expectations needed to meet the organization's goals. An involvement in a new market niche may bring the need to coordinate old products with a new one, to transfer expertise, and so on. One generally accepted cause of expectation shifts is organizational growth \cite{Blau1971,Haveman1993}. With growth comes specialization, and specialization consists of narrower role definitions. Therefore, other things being equal, we expect growing organizations to trend towards a greater degree of role specialization.


\subsection{Summary}

\begin{itemize}
\item Roles are \emph{sets of expectations of behaviour} placed on a person given their position, status, history, or preference.

\item Therefore, roles are \emph{social constructs}. There is no prototypical definition of what a certain role is.

\item \emph{Positions} (or \emph{titles}) are official statements of a person's goals and responsibilities. Positions are not roles, though they often carry the same labels.

\item We often use \emph{labels} to refer to roles, but some roles have ill-fitting or no labels in common use.

\item Expectations are partly set by our \emph{institutions}, that is, by the rules, norms, and culture prevalent in our society.

\item Performance that diverges from expectations, either positively or negatively, affects future expectations. Therefore, past performance in an organization causes a role \emph{evolution}.

\item Organizational growth brings new requirements into the organization, which also leads to role evolution. In particular, growth tends to cause role \emph{specialization}.
\end{itemize}



%%%%%%%%%%%%%%%%%%%%%%%%%%%%%%%%%%%%%%%%%%%%%%%%%%%%%%%%%%%%%%%%%%%%%%%%%%

\section{Study Design and Execution}

Based on the theoretical framework presented above, we performed an empirical study to improve our understanding of roles in software organizations, choosing to follow Yin's case study methodology \cite{Yin2003}. We executed an explanatory case study of two organizations, using two embedded units of analysis (software organizations and roles).

We formulated six research questions to guide our data collection and analysis. For each of them, we formulated a corresponding proposition of what we expected to find.

\textbf{\emph{RQ1:}} What common roles are there in software organizations? Of what expectations are they built from? Can we provide a simple classification of roles?

\emph{Proposition:} We do not expect to be able to provide a simple classification of roles. However, some expectations often come together, and these patterns may form a partial picture of roles in software development.

\textbf{\emph{RQ2:}} Do roles vary considerably between and within software organizations?

\emph{Proposition:} Roles vary considerably between software organizations and within them. The expectations placed on their members are different in significant ways.

\textbf{\emph{RQ3:}} What is the logic behind the particular clusters of expectations found in software organizations?

\emph{Proposition:} Expectations are clustered by institutional and organizational factors, but we do not know the specific reasons that drive this clustering.

\textbf{\emph{RQ4:}} Is there a significant divergence between roles and positions in software organizations? Why?

\emph{Proposition:} There is a significant divergence between roles and positions, and the reason is that \emph{position} describes at best a subset of the expectations shaping a role.

\textbf{\emph{RQ5:}} How do roles change over time in software groups?

\emph{Proposition:} There are two main drivers of role evolution: organizational growth, which leads to specialization, and divergences between expectation and performance, which lead to rearrangements of expectations.

\textbf{\emph{RQ6:}} What media do people enacting similar roles use to communicate with each other? What media do they use to communicate with others?

\emph{Proposition:} People enacting similar roles use similar communication media. They communicate with people enacting different but interfacing roles through other media, and these media act as a boundary object between them \cite{Bowker1999}.

We collected field data from two software organizations in the area of Victoria, BC, Canada. Our case selection was opportunistic: we wanted software development groups whose headquarters were in our vicinity. We negotiated access to both sites in return for feedback on their development practices.

The first author spent five weeks performing field observations (three weeks at the first site and two at the second), sitting at a desk in the organizations' development areas, attending several kinds of meetings, and exploring their issue tracking and project estimation systems.

The first author also interviewed people spanning most areas of each organization. He performed 36 interviews, most of them lasting approximately one hour. At both sites, the number of employees interviewed corresponded to about 50\% of the total number of group members at the time.



%%%%%%%%%%%%%%%%%%%%%%%%%%%%%%%%%%%%%%%%%%%%%%%%%%%%%%%%%%%%%%%%%%%%%%%%%

\section{Details from the Field Sites}

\emph{Gallium} was the first organization we studied.\footnote{For confidentiality reasons, the names of both organizations are fictional, and some details have been obfuscated.} It is a biotechnology software company that offers products commonly used by academic groups, pharmaceutical companies, and corporations in the sector. At the time of our observations, it employed about 45 people. Most of them worked at the Victoria office, though several managers and staff worked elsewhere in North America and Europe.

The development group was formed by 17 people, including the Director of Software Development,\footnote{All of the labels uppercased in this discussion refer to the \emph{positions} that these individuals hold, not to the roles they enact.} 2 Tech Leads, 3 Team Leads, 7 Developers, 3 QA Analysts, and a Buildmaster. It was divided into four teams, although one of them was being phased out and its members were transitioning onto another team. The largest team, Alpha, had nine people, and was in charge of maintaining and iterating on the organization's main product. The second largest, Beta, had five people, and was in charge of recreating and streamlining the workflow and user interface of the main product, without being burdened by its legacy code. The third team, Gamma, consisted of two people, and focused on generating custom scripts and solutions for individual customers, using the product's API. The fourth team, Delta, was wrapping up on its customer obligations for a secondary product. Everyone in the development group, except for its Director, worked in the same shared area: an open space with cubicles arranged in such a way as to demarcate smaller areas functioning as team rooms.

Several groups interacted with or supported the development group. There were Product, Program, and Project Managers, Customer Support Analysts, Field Specialists, Sales Executives, and others. Most of these people worked on a separate area in the company's offices.

Our observations began at an unusual point in the organization's history, as it had recently gone through a round of layoffs. While this was an unfortunate situation for the organization, it allowed us to explore in greater detail the mechanics of shifting expectations across the organizational structure and the employees' negotiation of the ``new normal'' at their offices.

The development group followed a lifecycle process loosely based on Scrum \cite{Schwaber2001}. The Alpha Team had a six-week release cycle, while Beta structured its work on shorter (two-week) cycles, but for internal purposes only, as it had not yet made any releases of its software. All teams had daily stand-up meetings, and the larger teams paid attention to Agile metrics such as burndown and velocity. The Team Leads functioned as the Scrum Masters, but the Product Owner was not defined consistently. The group also deviated from the simple backlogs and story card practices: their requirements were often specified in greater detail and they were derived from contractual obligations acquired by the sales team or from urgent issues identified by Field Specialists or Customer Support.

\emph{Cobalt} was the second organization we studied. In contrast to Gallium, it is publicly funded. It is a computing, data processing, and information storage unit servicing Canadian and international scientists. It is part of a research institute, but except for the sharing of support and hardware staff, it has a limited interaction with the rest of the institute. At the time of our observations, the group consisted of about 22 people, including visiting scientists that collaborated with the group's software development activities. All of Cobalt's employees worked at their Victoria offices.

The organization was divided into three subgroups: Software, Operations, and Science. The Software group had nine members, including the head of the group. Again, in contrast with Gallium, Cobalt's developers were not divided into teams; all may work for and request assistance from any of the projects of the group. However, every project had a Lead, and the group as a whole had five Subsystem Leads---a subsystem being an area of expertise for the group, such as Web Services or User Interface---that coordinate the work of those aspects for each project. There were no QA specialists in this group; developers were expected to test their work and that of their peers.

The Operations group had four employees, and it supported Software in coordinating, configuring, and ensuring the appropriate execution of its projects and processes. The Science group was formed by researchers; seven at the time of our observations. As the name indicates, they were tasked with doing scientific research, not with writing code. However, they had a significant say in the direction that the Software group took, and some of them developed script code. Finally, there are two staff employees in the group that do not form part of any team in particular: the Group Leader, and a Project Administrator.

The layout of the group's offices presented a mix of private offices and shared spaces with cubicles. Seniority seemed to be the main criterion in assigning private offices: more junior members of all three groups, and our observer, sat in the shared offices.

Our observations at Cobalt happened during a period of few changes or upsets. There had been two major restructures of the group (first, when the Software, Operations, and Science divisions were created; later, when the Software group created the Team Lead and Subsystem Lead structure), but they had both taken place years before our visit.

For months since and including the period of our observations, some group members had been experimenting with a few Agile practices. They introduced daily stand-up meetings in their projects, as well as continuous integration testing, but had only partial success in getting test-driven development or code reviews adopted in the group. And although they worked in one-week iterations, there was no release at the end of each cycle; the iterations were merely used as an opportunity to set the priorities for the next week.



%%%%%%%%%%%%%%%%%%%%%%%%%%%%%%%%%%%%%%%%%%%%%%%%%%%%%%%%%%%%%%%%%%%%%%%%%%

\section{Findings}

We present our findings organized by the research question they address.

\begingroup
\newcommand{\add}{\hspace{0pt}}

\definecolor{white}{rgb}{1,1,1}
\definecolor{mygray}{rgb}{0.7,0.7,0.7}
\definecolor{yourgray}{rgb}{0.4,0.4,0.4}
\definecolor{black}{rgb}{0.0,0.0,0.0}

\newdimen\qdx
\newdimen\qda
\newdimen\qdb
\newdimen\qd
\def\rrrr#1#2#3#4#5#6{\qd=#4 % length of bar for 1.0
\qdx=\qd\multiply\qdx by 5\divide\qdx by 4
\qda=\qd
\qdb=\qd
\multiply\qda by #1\divide\qda by #3\multiply\qdb by #2\divide\qdb by #3\advance\qdb by -\qda
    \leavevmode\hbox to \qdx{\hfil\vbox{%
    \hbox{\vrule\vbox{\hrule\hbox to 1\qd
            {\vrule depth0pt height#6 width \qda#5\vrule depth0pt height#6 width \qdb\hfill}\hrule}\vrule}
    }\hfil}}
\def\rrr#1#2#3#4{\rrrr{#1}{#2}{#3}{0.15cm}{#4}{1.5ex}}

\def\w{\rrr{0}{1}{1}{\color{white}}}
\def\l{\rrr{0}{1}{1}{\color{mygray}}}
\def\g{\rrr{0}{1}{1}{\color{yourgray}}}
\def\b{\rrr{0}{1}{1}{\color{black}}}
\def\0{\w}
\def\1{\l}
\def\2{\b}
\def\angle{60}

\begin{table*}[tb!]
\centering
%\footnotesize
%\scriptsize
\begin{tabular}{@{}l@{\hspace{-1.5cm}}r@{\hspace{5pt}}
c@{\hspace{2pt}}c@{\hspace{7pt}}
c@{\hspace{2pt}}c@{\hspace{7pt}}
c@{\hspace{2pt}}c@{\hspace{7pt}}
c@{\hspace{2pt}}c@{\hspace{7pt}}
c@{\hspace{2pt}}c@{\hspace{7pt}}
c@{\hspace{2pt}}c@{\hspace{7pt}}
c@{\hspace{2pt}}c@{\hspace{7pt}}
c@{\hspace{2pt}}c@{}}
\toprule
\vspace{1.4cm}\\
& Expectations 
&\multicolumn{2}{l}{\begin{rotate}{\angle}SW Directors\end{rotate} }
&\multicolumn{2}{l}{\begin{rotate}{\angle}Tech Leads\end{rotate} }
&\multicolumn{2}{l}{\begin{rotate}{\angle}Team Leads\end{rotate} }
&\multicolumn{2}{l}{\begin{rotate}{\angle}Developers\end{rotate} }
&\multicolumn{2}{l}{\begin{rotate}{\angle}Qa \& Ops\end{rotate} }
&\multicolumn{2}{l}{\begin{rotate}{\angle}Prod Mgmt\end{rotate} }
&\multicolumn{2}{l}{\begin{rotate}{\angle}Project Mgmt\end{rotate}} 
&\multicolumn{2}{l}{\begin{rotate}{\angle}Support \& Field\end{rotate}} \\
%& Expectations & \multicolumn{2}{l}{People} \\
\midrule
%&&& $p_1$ & $p_2$& $p_3$& $p_4$& $p_5$& $p_6$& $p_7$& $p_8$& $p_9$& $p_{10}$& $p_{11}$& $p_{12}$& $p_{13}$& $p_{14}$& $p_{15}$& $p_{16}$& $p_{17}$& $p_{18}$& $p_{19}$& $p_{20}$& $p_{21}$& $p_{22}$& $p_{23}$& $p_{24}$& $p_{25}$& $p_{26}$& $p_{27}$& $p_{28}$& $p_{29}$& $p_{30}$& $p_{31}$& $p_{32}$& $p_{33}$& $p_{34}$& $p_{35}$& $p_{36}$\\
%&&&p&p&p&p&p&p&p&p&p&p&p&p&p&p&p&p&p&p&p&p&p&p&p&p&p&p&p&p&p&p&p&p&p&p&p&p\\
%\midrule
\textbf{Technical} 
&Write bug-fix code& \0&\0&\1\1&\1\1&\1\2\0&\2\2&\1\1\1\0\1\0&\2\1&\0\0\0&\0&\0\0\0&\0&\0\0&\0&\0\0\0\0&\0\0 \\
\textbf{Core}
&Maintain legacy code& \0&\0&\0\1&\2\1&\2\0\0&\2\2&\2\0\2\0\2\2&\0\0&\0\0\0&\0&\0\0\0&\0&\0\0&\0&\0\0\0\0&\0\0  \\
&Write new-feature code& \0&\0&\2\2&\1\1&\1\2\0&\2\1&\1\2\0\0\0\1&\2\2&\0\0\0&\0&\0\0\0&\0&\0\0&\0&\0\0\0\0&\0\0 \\
&Write customization (or script) code& \0&\0&\0\0&\0\0&\0\0\2&\0\0&\0\0\0\2\0\0&\0\0&\0\0\0&\0&\0\0\0&\1&\0\0&\0&\0\2\0\0&\0\0\\
%& \\
%
%
\midrule
\textbf{Technical}
& Design user interface&\0&\0&\2\1&\0\0&\0\0\0&\0\0&\0\0\0\0\0\0&\0\2&\0\0\0&\0&\1\2\0&\0&\0\0&\0&\0\0\0\0&\0\0\\
\textbf{Periphery}
& Write documentation&\0&\0&\0\1&\0\1&\0\0\1&\0\1&\1\0\1\1\0\0&\0\0&\1\0\1&\0&\2\0\2&\0&\0\0&\0&\0\0\0\0&\0\0\\
&Write low-level (or unit) test code&\0&\0&\1\1&\0\1&\1\1\1&\1\1&\1\1\0\1\1\1&\0\0&\0\0\0&\0&\0\0\0&\0&\0\0&\0&\0\0\0\0&\0\0\\
& Write Build Code and prepare a release&\0&\0&\0\0&\0\0&\0\0\0&\1\0&\0\0\0\0\0\0&\0\0&\0\0\2&\0&\0\0\0&\0&\0\0&\0&\0\0\0\0&\0\0\\
& Write experimental (or exploratory) code&\0&\0&\1\2&\0\2&\0\0\0&\0\0&\0\0\0\0\2\0&\1\0&\0\0\0&\0&\0\0\0&\0&\0\0&\0&\0\0\0\0&\0\0\\
& Write high-level (or integration) test code&\0&\0&\0\0&\1\0&\0\0\0&\2\0&\0\0\0\0\0\0&\0\0&\2\1\0&\0&\0\0\0&\0&\0\0&\0&\0\0\0\0&\0\0\\
& Write configuration and implementation code&\0&\0&\0\0&\0\0&\0\0\1&\0\0&\0\0\0\0\0\0&\0\0&\0\0\0&\0&\0\0\0&\0&\0\0&\0&\0\2\0\0&\0\0\\
& Write and maintain tools to aid in development&\0&\0&\0\0&\0\1&\0\0\0&\1\0&\0\0\0\0\0\0&\0\0&\0\0\2&\1&\0\0\0&\0&\0\0&\0&\0\0\0\0&\0\0\\
%& \\
%
%
\midrule
\textbf{Technical}
& Design system architecture&\0&\0&\2\2&\2\2&\0\0\0&\1\1&\0\0\0\0\0\0&\0\1&\0\0\0&\0&\1\0\0&\0&\0\0&\0&\0\0\0\0&\0\0\\
\textbf{High level}
& Elicit and document requirements&\0&\0&\0\0&\0\1&\0\0\0&\1\1&\0\0\0\0\0\0&\0\0&\0\0\0&\0&\2\2\2&\0&\0\0&\0&\0\0\0\2&\1\2\\
& Determine if product meets quality goals&\1&\0&\1\1&\0\0&\1\0\0&\1\0&\0\1\0\0\0\0&\1\1&\2\2\1&\2&\0\2\0&\1&\0\0&\0&\0\0\0\0&\1\1\\
& Set roadmap for system (prioritize requirements)&\1&\0&\1\0&\0\1&\1\0\0&\2\0&\0\0\0\0\0\0&\0\0&\0\0\0&\0&\0\2\2&\2&\0\0&\0&\0\0\0\0&\2\1\\
& Provide domain context to non-domain experts (user proxy)&\0&\0&\0\0&\0\0&\0\0\0&\0\0&\0\0\0\0\0\0&\0\0&\0\0\0&\0&\0\2\0&\2&\0\0&\0&\0\2\0\2&\2\0\\
%& \\
%
%
\midrule
\textbf{Technical}
& Tackle the toughest technical issues and bugs&\0&\0&\0\2&\1\2&\2\0\2&\0\0&\1\0\0\0\2\0&\0\0&\0\0\0&\0&\0\0\0&\0&\0\0&\0&\0\0\0\0&\0\0\\
\textbf{Expertise}
& Provide historical perspective on design decisions&\0&\0&\2\1&\0\2&\0\0\0&\0\0&\2\0\2\0\0\0&\0\0&\0\1\0&\0&\2\0\0&\0&\0\0&\0&\0\0\0\0&\0\0\\
& Provide guidance (mentorship) to newcomers on technical issues&\0&\0&\1\0&\0\0&\0\0\2&\2\2&\2\0\0\0\0\0&\0\1&\0\0\0&\0&\0\0\0&\0&\0\0&\0&\0\0\0\0&\0\0\\
%& \\
%
%
\midrule
\textbf{Technical}
& Determine lifecycle process and practices&\2&\1&\1\0&\0\0&\1\0\1&\2\1&\0\0\0\0\0\0&\0\0&\0\0\0&\0&\0\0\0&\0&\1\2&\0&\0\0\0\0&\0\0\\
 \textbf{Management}
& Assign low-level tasks (tickets) to individuals&\0&\1&\1\0&\0\0&\2\0\1&\1\0&\0\0\0\0\0\0&\0\0&\0\0\0&\2&\0\0\1&\0&\1\1&\0&\1\2\1\0&\0\0\\
& Assign high-level tasks (projects) to individuals&\2&\2&\0\0&\0\1&\0\0\0&\0\0&\0\0\0\0\0\0&\0\0&\0\0\0&\0&\0\0\1&\0&\0\0&\0&\0\0\0\0&\0\0\\
& Coordinate a product release or implementation&\1&\1&\1\0&\1\1&\2\0\0&\2\0&\0\0\0\0\0\0&\0\0&\0\1\2&\2&\1\2\2&\0&\2\2&\1&\0\1\0\0&\0\0\\
& Provide estimates on remaining or potential tasks&\1&\1&\2\1&\1\0&\2\0\1&\2\0&\0\0\0\0\0\0&\0\0&\0\0\0&\0&\0\1\1&\0&\2\2&\1&\0\0\0\0&\0\0\\
& Connect an issue with the person with appropriate expertise&\1&\1&\0\0&\0\1&\2\0\0&\0\0&\0\0\0\0\0\0&\0\0&\0\0\0&\0&\0\0\0&\0&\1\1&\0&\1\2\2\0&\1\0\\
& Handle obstacles and interruptions for technical teams (PoC)&\1&\2&\0\0&\0\0&\2\1\1&\0\1&\0\0\0\0\0\0&\0\0&\0\0\0&\2&\0\0\0&\0&\2\2&\2&\0\0\0\0&\0\0\\
%& \\
%
%
\midrule
\textbf{Organizational}
& Receive and respond to user feedback&\0&\2&\0\0&\0\0&\0\1\0&\1\1&\0\0\0\0\0\0&\0\1&\0\0\0&\1&\0\0\0&\2&\2\2&\0&\2\2\2\0&\1\1\\
\textbf{External Interface}
& Obtain new users or sources of funding&\0&\2&\0\0&\0\0&\0\0\0&\0\0&\0\0\0\0\0\0&\0\0&\0\0\0&\0&\0\0\0&\1&\0\0&\0&\0\0\0\2&\0\0\\
& Provide domain legitimacy to organization&\0&\0&\0\0&\0\0&\0\0\0&\0\0&\0\0\0\0\0\0&\0\0&\0\0\0&\0&\0\2\0&\2&\0\0&\0&\0\0\0\0&\0\0\\
& Provide technical legitimacy to organization&\2&\1&\2\0&\0\2&\0\0\0&\0\1&\0\0\0\0\0\0&\0\0&\0\0\0&\1&\0\0\0&\0&\0\0&\0&\0\0\0\0&\0\0\\
& Advocate for the resolution of user/customer problems&\1&\2&\0\0&\0\0&\1\0\0&\0\0&\0\0\0\0\0\0&\0\0&\0\0\0&\1&\0\1\0&\0&\2\2&\0&\2\2\2\1&\2\2\\
& Represent the organization's interests to funders/customers&\0&\2&\0\0&\0\1&\0\1\0&\0\0&\0\0\0\0\0\0&\0\0&\0\0\0&\0&\0\1\0&\2&\0\0&\0&\0\0\0\1&\1\1\\
%& \\
%
%
\midrule
\textbf{Organizational}
& Deal with personnel issues&\2&\2&\0\1&\0\0&\1\1\0&\0\0&\0\0\0\0\0\0&\0\0&\0\0\0&\1&\0\0\0&\1&\0\0&\2&\0\0\1\0&\0\0\\
\textbf{Management}
& Deal with otherwise unattended issues&\1&\1&\0\0&\0\1&\1\0\0&\0\0&\0\0\0\0\0\0&\0\0&\0\2\0&\1&\0\1\0&\1&\0\0&\2&\0\0\0\0&\0\0\\
& Set up and modify organizational structure&\2&\2&\0\0&\0\1&\0\0\0&\0\0&\0\0\0\0\0\0&\0\0&\0\0\0&\1&\0\0\1&\1&\0\0&\0&\0\0\1\0&\0\0\\
& Communicate organizational status and news&\2&\2&\0\0&\0\0&\1\0\1&\0\0&\0\0\0\0\0\0&\0\0&\0\0\0&\1&\0\0\0&\0&\1\1&\1&\0\0\1\0&\0\0\\
& Deal with financial, clerical, or system administration issues&\0&\1&\0\0&\0\0&\0\0\0&\0\0&\0\0\0\0\0\0&\0\0&\0\0\0&\1&\0\0\0&\0&\0\0&\2&\0\0\0\0&\0\0\\
%& \\
%
%
\midrule
\textbf{Organizational}
& Determine long-term vision for products&\1&\2&\1\0&\0\2&\0\0\0&\0\0&\0\0\0\0\0\0&\0\0&\0\0\0&\0&\0\2\0&\2&\0\0&\0&\0\0\0\0&\0\0\\
\textbf{Strategic}
& Determine emphasis on new market areas&\0&\1&\0\0&\0\1&\0\0\0&\0\0&\0\0\0\0\0\0&\0\0&\0\0\0&\0&\0\2\0&\1&\0\0&\0&\0\0\0\0&\0\0\\
%& \\
\midrule
& Groups 
& A & B
& A & B
& A & B
& A & B 
& A & B
& A & B
& A & B
& A & B\\
\midrule
\multicolumn{18}{c}{\2: something something \1: something something \0: something something}\\
\end{tabular}
\caption{Alle meine Entchen schwimmen auf dem See, schwimmen auf dem See, K\"{o}pfchen in das Wasser, Schw\"{a}nzchen in die H\"{o}h'.
Alle meine T\"{a}ubchen
gurren auf dem Dach,
gurren auf dem Dach,
fliegt eins in die L\"{u}fte,
fliegen alle nach.
Alle meine H\"{u}hner
scharren in dem Stroh,
scharren in dem Stroh,
finden sie ein K\"{o}rnchen,
sind sie alle froh.
Alle meine G\"{a}nschen
watscheln durch den Grund,
watscheln durch den Grund,
suchen in dem T\"{u}mpel,
werden kugelrund.}
\end{table*}
\endgroup


\subsection{Role Classifications and Expectations (RQ1)}

There was no simple role classification in the organizations we studied. However, it appears that some patterns in role definitions would emerge with further studies. We can see the beginnings of such patterns in Table \ref{tableExpectations}, which shows the expectations placed on the people we interviewed. The columns in the table represent our interviewees and the rows represent common or significant expectations. The reader should note that the list of expectations is necessarily incomplete for the general case. It is also somewhat subjective, as expectations were not always clearly articulated, and our participants did not agree on their expectation assignments in every case.

One pattern concerns the technical core of the organizations' products. Work on this area was reserved almost exclusively for developers and their technical and team leads. However, the expectations placed on leads were more spread out: technical leads focused more on architecture and design issues, whereas team leads emphasized coordination activities. There were also purely organizational roles that do not need any domain or technical expertise. But in both organizations some key people had more than one kind of expertise (technical, domain, and organizational), and these people seemed vital for the functioning of the organization as they transitioned between boundaries with ease. A representative example is a Gallium Field Specialist (the second column in the Support \& Field cluster), who straddled all three kinds of expertise. Several people in different areas pointed out his relevance; in the words of one informant, ``the company depends on [him] so much he should have armed guards behind him taking him to and from work.''

Our data shows several other expectation assignment patterns, but for space reasons, we only point out three. First, all the system architecture expectations were placed on people still working on the technical core, writing new feature and bug fix code. This is in contrast to our anecdotal experiences in larger organizations. Second, group members that provided domain legitimacy or credibility were critical at both organizations, and they were expected to influence product development at a high level, but not to tinker with implementation details. And third, some expectations, such as writing unit tests and documentation, were weakly spread out through the technical side of the organizations. Everyone expected developers to ``play nice'' by testing and documenting their own code, but it was culturally acceptable to ignore these expectations if the urgency of the situation demands it---as it often does.


\subsection{Role Variation (RQ2)}

As Table \ref{tableExpectations} suggests, we found a considerable evidence of variation in roles between and within organizations. In fact, it was difficult to find people in either site with the same role in significant detail. We offer several examples of this phenomenon, as well as a key counterexample.

\emph{Product Management at Gallium:} There were three people with the role of steering the organization's products, prioritizing its backlog, and coordinating with customers: the two Product Managers and the Product Design Lead. The latter documented the requirements of new functionality, provided a historical perspective on design decisions, and helped shape the organizational structure. Of the two Product Managers, one had an inside-looking approach, whereas the other one had an outside-looking approach: he had a Ph.D.\ in the organization's area of expertise, and his presence at client meetings, trade shows, and conferences legitimized the organization. He also understood how the users would interact with the product and helped design the user interface to bring it closer to the users' mental model. Other people in the organization also performed some Product Management activities unofficially: Customer Services Representatives and Field Specialists often persuade developers to focus on the features that cause them the most pain, for example. In sum, although the three employees above have similar goals, the expectations placed on them vary considerably, are not interchangeable, and are not their exclusive prerogative.

\emph{Team Leading at Gallium and Cobalt:} Every team in the sites we studied had a Team Lead, but this leadership role was tinged by the characteristics of the team in question. In Cobalt, team membership is fluid---every developer may be expected to work on any project, according to need---and some Team Leads performed coordination tasks, others had a dormant leadership position, waiting for new issues to arise, and yet others were more like ``first responders'' than coordinators, requesting assistance only if their work load became excessive. In Gallium, there were three Team Leads at the time of our observations. Their roles were quite different. The Team Lead of the Alpha Team would spend about half of his time developing software, and the other half in meetings, performing estimation tasks, and analyzing the performance of his team. A second Team Lead was in charge of a product about to be discontinued, but that responsibility took a small part of his time. He was mostly associated with the Beta Team, but since that team had an abundance of ``leads'' (it included the two official Tech Leads of the organization), his coordination and reporting expectations were minimal. The third Team Lead had one other team member, a newcomer to the organization, and his leadership expectations consisted mostly of mentoring and training rather than coordination and monitoring.

\emph{Developers at Gallium and Cobalt:} The quintessential role in the software development literature was one of the hardest to pin down. A developer, of course, is expected to write code, but what else? There is a multitude of answers to that question: some developers are expected to perform code reviews, some to provide architectural input, some to carry their own requirements analysis, some to socialize, some to interface with customers. None of the developers in our sample were expected to perform all of the above.

\emph{``Glue'' role, Gallium and Cobalt:} In both sites we found at least one person who others referred to, with more or less these words, as ``the glue that holds this place together.'' They generally meant a person that would do all sorts of things, little or large, in a variety of contexts, stuff that falls through the cracks of the formal organization. But the stuff that falls through the cracks was different in both sites, and therefore, the expectations were different. In Cobalt, much of this referred to sheltering the organization from bureaucratic processes and constraints. In Gallium, it partly meant helping create and shape new and needed groups, and partly to perform activities that nobody else was tasked with doing, such as some kinds of documentation.

\emph{Project Management at Gallium:} These Project Managers were a counterexample to our findings, to some extent. Although there were distinctions between them (one had a Project Management Leader role, another created templates and instituted new processes, and all, including a third and more junior Project Manager, implemented them), the similarities in their roles seemed to outweigh the differences. The main reason seemed to be that they brought a fairly specific and standardized way of doing things with them, in the form of a Project Management Body of Knowledge and certification. In other words, the institutionalization of their role was strong. They also strived to define their role with precision, and to enforce their definition in the rest of the organization. No other role that we examined had the emphasis on role definition and consistency that the Gallium Project Managers had.


\subsection{Expectation Clustering (RQ3)}

We found that expectations are grouped in roles based on a combination of \emph{institutional}, \emph{historical}, \emph{structural}, and \emph{personal} factors, and that these factors interplay and provide feedback to each other. 

\emph{Institutional factors:} Although the software industry's culture still leaves plenty of room for role variability, it helps shape role definitions. We have several examples of this, but the widest in scope is when Gallium, after struggling for a few years as a self-funded start-up, acquired a considerable amount of venture capital. It quickly expanded and distributed its labour in predictable ways: it created several groups (Product Management, Quality Assurance, Sales, Customer Support, Human Resources) and filled them with newcomers. Before this influx, the division of labour was flexible and organic: most organization members did a bit of everything. After the influx, role definitions shifted towards a more stereotypical differentiation.

\emph{Historical factors:} Previous events in the history of the organization shape its current grouping of expectations, in the same way that alterations early in the life of other organic systems shape their later structure. Some developers that have been part of the organization for a long time are not just developers, they are also mentors, historians, and political players, despite what the organizational chart says about them. This affects the expectations placed on them. During the Gallium downsizing, some people were demoted to keep the company afloat. At some point, one of them had been the head of the QA department (which was dispersed) and, later, the Program Manager. At the time of our observations he was a QA Analyst, yet he was still expected to be involved in strategic activities that did not match the institutional view of what a QA Analyst is supposed to do.

\emph{Structural factors:} A person's position in the organizational structure gives her a certain power and awareness that others may not have. Her structural position leads others to expect from her certain behaviours: to play a bridging, directing, or even confrontational role. Many of the expectations placed on the head of the Operations group at Cobalt spawned from his structural position. He was expected to provide news of the operational status of Cobalt's services, to deal with and shelter the Software group from system administration concerns, and to control the gateway that determines whether a product is ready to be released or not---that is, to have final say on the operational quality of the product. These expectations come from his position in the organization. It would be hard for someone in a different position to satisfy them.

\emph{Personal factors:} A person's background, preferences, and personality shapes the expectations she receives from others. Perhaps her background is deficient for the fulfillment of her initial role, and others adapt by expecting different or lesser things from her. Alternatively, her traits (curiosity, proactivity, a talent) may lead her to take on responsibilities that would not otherwise be placed on her given institutional, historical, or structural factors. For example, the Project Administrator at Cobalt was hired to perform clerical tasks, but his own initiative led him to gradually take more and greater responsibilities. The organization learned to be dependent on such behaviour. When he temporarily left the group to fulfill a position elsewhere in the Institute, Cobalt brought in a temporary replacement. Projects suffered quickly and significantly as the replacement only focused on the expectations that were officially placed on him. We should also note that at the time of our observations, the Project Administrator position was undergoing an official redefinition in order to match real expectations. In this way, personal factors were leading to historical and structural changes, so that future enactors of the role played by the Project Administrator would be affected by them.


\subsection{Divergence between Roles and Positions (RQ4)}

We found a divergence between people's positions and the expectations placed on them. There is an overlap, however, and it is worth exploring first. One can reasonably expect a Developer to write code of some kind, a QA Analyst to be concerned with software quality in one way or another, and a Team Lead to monitor and report on the team's progress. Table \ref{tableExpectations} provides an abstracted view of these patterns, which also hold throughout our more detailed qualitative data. We believe that this partial agreement comes from the fact that a person's position is, first, a statement about her placement in the organizational \emph{structure}, and second, a partial statement of the \emph{institutional} expectations placed on her. That is, those in the organization with the requisite authority give her a label that approximates their expectations (though perhaps not those of the rest of the organization) and place her in a position in the structure that should help her satisfy them.

However, a person's position has limited power in defining their role. As we have seen above, the expectations placed on a person also come from historical and personal factors. An example from our observations is one of the Gallium Product Managers. He was hired to steer and coordinate the development of the company's products, but he had no experience in software development. He thought that his main asset was his understanding of and his contacts in the products' domain, so that he was in the organization as the ``voice of the customer.'' In his early days in the organization, he struggled to understand the level of control he was expected to have. Over time, he adjusted his behaviour, but he continued to make use of his background, connections, and domain intuition to give his role a distinctly different flavour than those of others with similar positions.


\subsection{Role Evolution (RQ5)}

We proposed that roles evolve for two reasons: organizational growth, and mismatches between expectation and performance. Our data provide support for both reasons.

Regarding organizational growth, it has been recognized as a factor in specialization for a long time, and our data are far from surprising. In fact, it should be clear from pure argument that growth drives role evolution. The smallest organization is formed by a single person, and that person should perform all the activities required for the functioning of the organization, from sales to debugging. As the organization obtains more members, labour begins to be divided, so that upon reaching even a handful of members, it is rare for a software group to perform undifferentiated labour. Furthermore, growth carries a coordination weight: larger organizations need to assign some of their members the role of coordinating the work of others, or of interfacing between groups. That is, larger organizations have needs that are caused by their size. With organizations that are similar to those we observed, some degree of specialization is well underway. We expect that, should we perform this study in a larger organization, we would discover further role specialization.

Growth-driven role specialization can be resisted and overturned, as some Agile proposals suggest. In this sense, however, the most interesting event in our observations was an attempt to ``break silos'' at Cobalt, which was successful for its stated goals, but \emph{increased} role differentiation. The silos that the group wanted to break were project-based. They discovered that assigning one person per project introduced vulnerabilities (when something went wrong in a project, only one person could intervene) and rework (as many projects consisted of similar modules that had to be developed independently). To break these silos, Cobalt assigned five of its senior developers the position of Subsystem Leads, so that each of them would be the first point of contact for matters concerning their subsystem. The end result was a greater differentiation between the roles of the five Subsystem Leads (previously simply Developers), but also a perceived decrease on the level of isolation, as projects begun to cross-pollinate.

The other important factor driving role evolution was a mismatch between expectations and behaviour. The fact that expectations are placed does not mean that they are satisfied. There are many potential reasons for their lack of satisfaction: for instance, the person upon whom they are placed may not know or understand the expectation, it could be beyond her means, or it could be in conflict with other expectations. Therefore, an assignment of expectations that works in theory may not work in practice. Similarly, an assignment that does \emph{not} work in theory may work in practice, due to the initiative or motivation of people who contribute beyond their expectations. However, in both cases, with time, organization members grow to expect the typical performance of others, and a new expectation (and hence, a new role) takes hold.

Several examples from our data above illustrate this phenomenon and we offer one more here. The coordinator of the Science group at Cobalt was brought in to do research with the data that Cobalt serviced in a ``consultative role.'' As time progressed, and without being contractually obliged nor pressured by his peers, he begun taking other responsibilities: providing feedback on how users would use some service features, being a contact for scientists outside the group (which lends more domain legitimacy to the organization), and helping to steer the group's direction. In many ways, his role begun to look more like that of the Gallium Product Manager we discussed above, without being named or being perceived as a Product Manager by anybody in the organization. As a result, he now plays an essential role in the development of the group's products and in the maintenance of its relations with its users, a role that the rest of the organization depends on for its good functioning.



\subsection{Roles and Communication Media (RQ6)}

We expected to find that people enacting similar roles use similar communication media, and that they communicate with people enacting interfacing roles through other media. Although we found support for this claim, a biasing factor was the prevalence of face-to-face interactions in both of our field sites. Regardless of role, people communicated face-to-face regularly. Stand-up meetings, walks across the hall, and group meetings were commonplace.

Regarding other media, we found role-based differences in medium selection and use. The broadest medium---the one that had the greatest use among the most kinds of roles---was the issue tracking system. People with expectations regarding code production, quality assurance, customer service, field implementation, product management, and performance monitoring all used the issue tracking system to some extent. Its use was not uniform: some group members used it to emit messages, others were mostly listeners, or they were more interested in the aggregate activity of the team. This finding mirrors the observation of De Souza \emph{et al.}\ regarding issue tracking systems as boundary objects \cite{deSouza2003}.

The use of other media was more localized. In Gallium, IRC use was common among some developer teams: each team had one channel, and these were not frequented by any non-developers. Customer Support staff used Google Documents to keep track of its cases, and to provide an ongoing documentation of recent events and requests for support. Email and phone were more heavily used in the management side of the organization: developers and QA analysts rarely made or received phone calls, and they obtained most of the information they needed about their projects through the issue tracking system.



%%%%%%%%%%%%%%%%%%%%%%%%%%%%%%%%%%%%%%%%%%%%%%%%%%%%%%%%%%%%%%%%%%%%%%%%%%%%%%%

\section{Discussion}

We explored the sociological and organizational literature to generate a theoretical framework of roles and expectations. We used that framework to study several research questions regarding work in software organizations.

We found that roles, defined as sets of expectations, were inconsistent within and between the organizations we studied. Roles are socially constructed; part of the construction occurs at the cultural or institutional level, and part occurs at the local, team level, and is caused by historical, structural, and personal factors. Although we can make some reasonable assumptions about the work of a software professional based on his label or official position, our assumptions will only take us so far: their responsibilities vary considerably, and treating them as archetypical professionals (``developers,'' ``testers,'' and so on) dismisses a large part of what they bring or fail to bring to their job.

We formulated a list of expectations that were common or significant in the two sites we observed. We do not think that our list is exhaustive, but we do think that it represents expectations of concern in most software organizations, and that formulating such lists and analyzing how expectations map to organization members gives us a good understanding of role definitions and interactions for software development.

We also explored role evolution and found that it is driven by organizational growth, and by mismatches between expectations and behaviour. Finally, we found that people enacting different roles tend to use different media to communicate, and that face-to-face interaction and issue-tracking notifications were the most widely used media in our sites across roles.


\subsection{Implications}

\emph{Research:} First, the fact that role definitions are not institutionally uniform and respond to local (historical, structural, personal) factors should help clarify common misunderstandings and assumptions made in research. A stereotypical developer, team lead, or tester, is rarely found in practice. Instead, thinking in terms of tasks and expectation groupings will help us ground our studies to software development in the real world.

Second, thinking in terms of sets of expectations, rather than positions, clears the path for an analysis of good patterns of expectation pairings (expectations that make good sense, organizationally, to be placed on the same individual). This in turn should lead to better role definitions, and to the study of bottlenecks and deficiencies in organizational structure. For instance, the discussion of the proper ratio of developers to testers, which is moot for teams where there is no QA and developers do all the testing, can be explored in terms of the proper ratio of people expected to write code to people expected to write integration tests or to determine the level of quality of the product.

Third, the observation that media use varies depending on the kinds of expectations placed on the user complements the earlier finding that repositories are incomplete and often misleading \cite{Aranda2009}. If researchers focus on a single repository, they are likely to systematically miss important areas of coordination: they will be blind to some people and expectations. Depending on their research questions, this may be a serious concern. Judging by our two cases, mining issue tracking systems should yield the greatest payoff, although even then, mining algorithms would miss the people that mostly listen or pay attention to the aggregate activity.

\emph{Practice:} Our data confirm, in the software development domain, the old sociological finding that the role specialization trend as an organization grows is hard to fight. This is not necessarily bad, but teams of would-be generalists (which Agile techniques often propose) need to calculate the long-term impacts of seemingly necessary short-term specializations. A role that makes sense in the present to deal with a dysfunction may spin off to become a compartmentalized department later on, with its own political, budgetary, and organizational goals. This may not be in the interest of the current members of the organization.

Additionally, we found that practitioners may not have an awareness of the full scope of their peers' activities, and that this lack of awareness leads them to misunderstand the benefits that they bring to the organization if they make judgments based on labels. It may be the case for them, as it was in the two groups we studied, that people nominally tasked with minor matters actually help the team in deeper, more complex ways, and that their formal position does not match the expectations placed on them. In these cases, a reductionism to position labels is harmful.


\subsection{Threats to Validity}

The two organizations we studied were small (less than fifty members), geographically co-located for the most part, and North American. Therefore, the generalizability of our findings might be questioned. However, most of our findings are supported by a theoretical framework that is not built upon these particular characteristics, but is instead general and interdisciplinary. We expect that some of the particular expectation pairings will vary in other groups, but we also expect the underlying rationale and observations to hold across them.

During our observations, Gallium underwent some role transitions as it struggled to rearrange itself after a round of layoffs. We kept in mind the unstable state of this organization in our data analysis, and we believe that it contributed to our understanding of role evolution and mismatches in expectation assignment.

Throughout this paper, we defined role as a ``set of expectations.'' However, an ``expectation'' is a subjective concept: we expect many things from many people, and an objective articulation of these expectations may never be quite right, missing some attributes or formulating an expectation at the wrong level of detail. We acknowledge this threat, but it is a necessary evil. We hope that with more research in the area, a consensus will emerge on the right level of analysis for expectations in software organizations.



%%%%%%%%%%%%%%%%%%%%%%%%%%%%%%%%%%%%%%%%%%%%%%%%%%%%%%%%%%%%%%%%%%%%%%%%%%%%%%%%%

\section{Conclusion}

In this paper we presented a theoretical framework to make sense of roles and expectations in software development organizations, and we reported on a case study that used this framework to produce insights into coordination and the division of labour for software professionals.

The framework and our findings generate several further questions, which we plan to address in future work. First, there is the issue of the right level of granularity to articulate expectations in software development. Second, an analysis of commonly-found and commonly-useful patterns in expectation clustering. Third, an analysis of  appropriate ratios of expectation assignment in software groups. And fourth, an exploration of the differences in specialization and expectation clustering in organizations that differ significantly from those we observed. We hope others in the research community join us in addressing these questions.



%%%%%%%%%%%%%%%%%%%%%%%%%%%%%%%%%%%%%%%%%%%%%%%%%%%%%%%%%%%%%%%%%%%%%%%%%%%%%%%%%

\section*{Acknowledgments}

We thank the participants of our study for their time, and Cassandra Petrachenko and Jamie Starke for their comments.

\bibliographystyle{IEEEtran}
\bibliography{biblio}

\end{document}


